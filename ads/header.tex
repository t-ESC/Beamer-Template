%!TEX root = ../template.tex
\documentclass[
    12pt,				% Schriftgroesse
    % headheight = 33pt,	% Höhe der Kopfzeile
    % footheight = 16pt,	% Höhe der Fusszeile
    % headinclude=false,	% Kopfzeile nicht in den Satzspiegel einbeziehen
    % footinclude=false,	% Fußzeile nicht in den Satzspiegel einbeziehen
]{beamer}
% Disable Beamer Navigation Bar
\setbeamertemplate{navigation symbols}{}


%%%%%%% Package Includes %%%%%%%
\usepackage{tikz}
\usepackage{graphicx}
\graphicspath{{../images/}}
\usepackage{setspace}
\usepackage{longtable}
\usepackage{multirow}
\usepackage[ngerman]{babel}
\usepackage{xstring}

%%%%%%% Commands %%%%%%%
\newcommand{\einstellung}[1]{%
    \expandafter\newcommand\csname #1\endcsname{}
    \expandafter\newcommand\csname setze#1\endcsname[1]{\expandafter\renewcommand\csname#1\endcsname{##1}}
} %Einstellungscommand

\newcommand{\langstr}[1]{\einstellung{lang#1}} % Sprache aus Einstellungen laden

\newcommand{\citem}[1]{\item[\texttt{#1}]} % Code-Item für description-Liste

\newcommand{\todo}[1]{\textit{\textcolor{red}{TODO: #1}}} % Todo-Item

\newcommand{\ladefarben}{%
	\definecolor{LinkColor}{HTML}{00007A}
	\definecolor{ListingBackground}{HTML}{FCF7DE}
} % Farben (Angabe in HTML-Notation mit großen Buchstaben)

%% Programmiersprachen Highlighting (Listings)
\newcommand{\listingsettings}{%
	\lstset{%
		language=C++,			% Standardsprache des Quellcodes
		%numbers=left,			% Zeilennummern links
		%stepnumber=1,			% Jede Zeile nummerieren.
		%numbersep=5pt,			% 5pt Abstand zum Quellcode
		%numberstyle=\tiny,		% Zeichengrösse 'tiny' für die Nummern.
		breaklines=true,		% Zeilen umbrechen wenn notwendig.
		breakautoindent=true,	% Nach dem Zeilenumbruch Zeile einrücken.
		postbreak=\space,		% Bei Leerzeichen umbrechen.
		tabsize=2,				% Tabulatorgrösse 2
		basicstyle=\ttfamily\footnotesize, % Nichtproportionale Schrift, klein für den Quellcode
		showspaces=false,		% Leerzeichen nicht anzeigen.
		showstringspaces=false,	% Leerzeichen auch in Strings ('') nicht anzeigen.
		extendedchars=true,		% Alle Zeichen vom Latin1 Zeichensatz anzeigen.
		captionpos=b,			% sets the caption-position to bottom
		%backgroundcolor=\color{ListingBackground}, % Hintergrundfarbe des Quellcodes setzen.
		xleftmargin=0pt,		% Rand links
		xrightmargin=0pt,		% Rand rechts
		frame=single,			% Rahmen an
		frameround=ffff,
		rulecolor=\color{darkgray},	% Rahmenfarbe
		%fillcolor=\color{ListingBackground},
		keywordstyle=\color[rgb]{0.133,0.133,0.6},
		commentstyle=\color[rgb]{0.133,0.545,0.133},
		stringstyle=\color[rgb]{0.627,0.126,0.941},
    aboveskip=1.5em,
	}
}


%%%%%%% Flags %%%%%%%
% Flag für Inhaltsverzeichnis
\newif\ifinhalt
\inhaltfalse

% Flag für die Selbstständigkeitserklärung, Default: true
\newif\ifselbsterkl
\selbsterklfalse

% Flag für roten Vertraulichkeitspunkt, default: false
\newif\ifreddot
\reddotfalse

% Flag für gelben Vertraulichkeitspunkt, default: false
\newif\ifyellowdot
\yellowdotfalse


% Flag für Literaturverzeichnis
\newif\ifliteratur
\literaturfalse

\einstellung{martrikelnr}
\einstellung{titel}
\einstellung{kurs}
\einstellung{datumAnfang}
\einstellung{datumAbgabe}
\einstellung{firma}
\einstellung{firmenort}
\einstellung{abgabeort}
\einstellung{abschluss}
\einstellung{studiengang}
\einstellung{dhbw}
\einstellung{betreuer}
\einstellung{gutachter}
\einstellung{zeitraum}
\einstellung{arbeit}
\einstellung{autor}
\einstellung{sprache}
\einstellung{schriftart}
\einstellung{kapitelabstand}
\einstellung{spaltenabstand}
\einstellung{zeilenabstand}
\einstellung{zitierstil}
\einstellung{selbsterkl}
\einstellung{semester}
\einstellung{studienrichtung}
\einstellung{jahrgang}
\einstellung{abteilung}
\einstellung{standort}
\einstellung{formelbeschreibung} % verfügbare Einstellungen
% !TeX root = template.tex
%%%%%%%%%%%%%%%%%%%%%%%%%%%%%%%%%%%% Sprache %%%%%%%%%%%%%%%%%%%%%%%%%%%%%%%%%%%
%% Aktuell sind Deutsch und Englisch unterstützt.
%% Es werden nicht nur alle vom Dokument erzeugten Texte in
%% der entsprechenden Sprache angezeigt, sondern auch weitere
%% Aspekte angepasst, wie z.B. die Anführungszeichen und
%% Datumsformate.
\setzesprache{de} % de oder en


%%%%%%%%%%%%%%%%%%%%%%%%%%%%%%%%%%% Angaben  %%%%%%%%%%%%%%%%%%%%%%%%%%%%%%%%%%%
%% Die meisten der folgenden Daten werden auf dem
%% Deckblatt angezeigt, einige auch im weiteren Verlauf
%% des Dokuments.
\setzemartrikelnr{-setzemartrikelnr-}
\setzekurs{-setzekurs-}
\setzetitel{-setzetitel-}
\setzedatumAnfang{-setzedatumAnfang-}
\setzedatumAbgabe{-setzedatumAbgabe-}
\setzefirma{-setzefirma-}
\setzefirmenort{-setzefirmenort-}
\setzeabgabeort{-setzeabgabeort-}
\setzeabschluss{-setzeabschluss-}
\setzestudiengang{-setzestudiengang-}
\setzedhbw{-setzedhbw-}
\setzebetreuer{-setzebetreuer-}
\setzegutachter{-setzegutachter-}
\setzezeitraum{-setzezeitraum-}
\setzearbeit{-setzearbeit-}
\setzeautor{-setzeautor-}
\setzesemester{-setzesemester-}
\setzestudienrichtung{-setzestudienrichtung-}
\setzejahrgang{-setzejahrgang-}
\setzeabteilung{-setzeabteilung-}
\setzestandort{-setzestandort-}

\inhalttrue                 % auskommentieren oder ändern zu \inhaltfalse, falls kein Inhaltsverzeichnis eingefügt werden soll
% \abkverztrue                % auskommentieren oder ändern zu \abkverzfalse, wenn kein Abkürzungsverzeichnis benötigt wird
% \literaturtrue              % auskommentieren oder ändern zu \literaturfalse, wenn kein Literaturverzeichnis gewünscht ist (\appendixtrue muss gesetzt sein!)
% \selbsterkltrue             % auskommentieren oder ändern zu \selbsterklfalse, wenn keine Selbstständigkeitserklärung benötigt wird
% \firmatrue                  % auskommentieren oder ändern zu \selbsterklfalse, wenn kein Firmenlogo benötigt wird

% Angabe des roten/gelben/grünen Punktes auf dem Titelblatt zur Kennzeichnung der Vertraulichkeitsstufe.
% Mögliche Angaben sind \yellowdottrue, \reddottrue. Werden beide angegeben, wird der rote Punkt gezeichnet.
% Wird keines der Kommandos angegeben, wird der grüner Punkt gezeichnet
% \reddottrue
% \yellowdottrue


%%%%%%%%%%%%%%%%%%%%%%%%%%%% Literaturverzeichnis %%%%%%%%%%%%%%%%%%%%%%%%%%%%%%
%% Bei Fehlern während der Verarbeitung bitte in ads/header.tex bei der
%% Einbindung des Pakets biblatex (ungefähr ab Zeile 110,
%% einmal für jede Sprache), biber in bibtex ändern.
\newcommand{\ladeliteratur}{%
    \addbibresource{bibliographie.bibtex}
}

%% Zitierstil
%% siehe: http://ctan.mirrorcatalogs.com/macros/latex/contrib/biblatex/doc/biblatex.pdf (3.3.1 Citation Styles)
%% mögliche Werte z.B numeric-comp, alphabetic, authoryear
\setzezitierstil{ieee}
%%%%%%%%%%%%%%%%%%%%%%%%%%%%%%%%%%%%%%%%%%%%%%%%%%%%%%%%%%%%%%%%%%%%%%%%%%%%%%%%

%%%%%%%%%%%%%%%%%%%%%%%%%%%%%%%%% Layout %%%%%%%%%%%%%%%%%%%%%%%%%%%%%%%%%%%%%%%
%% Verschiedene Schriftarten
% laut nag Warnung: palatino obsolete, use mathpazo, helvet (option scaled=.95), courier instead
\setzeschriftart{lmodern} % palatino oder goudysans, lmodern, libertine

%% Abstand vor Kapitelüberschriften zum oberen Seitenrand
\setzekapitelabstand{20pt}

%% Spaltenabstand
\setzespaltenabstand{10pt}
%%Zeilenabstand innerhalb einer Tabelle
\setzezeilenabstand{1.5}
%%%%%%%%%%%%%%%%%%%%%%%%%%%%%%%%%%%%%%%%%%%%%%%%%%%%%%%%%%%%%%%%%%%%%%%%%%%%%%%% % lese Einstellungen


\newcommand{\iflang}[2]{%
  \IfStrEq{\sprache}{#1}{#2}{}
}

\langstr{abkverz}
\langstr{anhang}
\langstr{glossar}
\langstr{deckblattabschlusshinleitung}
\langstr{artikelstudiengang}
\langstr{studiengang}
\langstr{anderdh}
\langstr{von}
\langstr{dbbearbeitungszeit}
\langstr{dbmatriknr}
\langstr{dbkurs}
\langstr{dbfirma}
\langstr{dbbetreuer}
\langstr{dbgutachter}
\langstr{sperrvermerk}
\langstr{erklaerung}
\langstr{abstract}
\langstr{listingname}
\langstr{listlistingname}
\langstr{listingautorefname}
\langstr{selbsterkl}
\langstr{formelsammlung}
\langstr{kopfz}
\langstr{fussz}
\langstr{seite}
\langstr{seitevon}
\langstr{stand}
\langstr{formelgroeverz}
\langstr{toc} % verfügbare Strings
\input{lang/\sprache} % Übersetzung einlesen

% Einstellung der Sprache des Paketes Babel und der Verzeichnisüberschriften

\PassOptionsToPackage{english, ngerman}{babel}

\iflang{de}{
    \usepackage{babel}
    \selectlanguage{ngerman}
}
\iflang{en}{
    \usepackage{babel}
    \selectlanguage{english}
}

%%%%%% Configuration %%%%%
\usepackage{\schriftart}
\ladefarben


% Titel, Autor und Datum
\title{\titel}
\author{\autor}
\date{\datum}

% % PDF Einstellungen
% \usepackage[%
%     pdftitle={\titel},
%     pdfauthor={\autor},
%     pdfsubject={\arbeit},
%     pdfcreator={pdflatex, LaTeX with KOMA-Script},
%     pdfpagemode=UseOutlines, 		% Beim Oeffnen Inhaltsverzeichnis anzeigen
%     pdfdisplaydoctitle=true, 		% Dokumenttitel statt Dateiname anzeigen.
%     pdflang={\sprache}, 			% Sprache des Dokuments.
% ]{hyperref}

% % (Farb-)einstellungen für die Links im PDF
% \hypersetup{%
%     colorlinks=true, 		% Aktivieren von farbigen Links im Dokument
%     linkcolor=black, 	    % Farbe festlegen
%     citecolor=LinkColor,
%     filecolor=LinkColor,
%     menucolor=LinkColor,
%     urlcolor=LinkColor,
%     %linktocpage=true, 		% Nicht der Text sondern die Seitenzahlen in Verzeichnissen klickbar
%     linktoc=all,            % Seitenzahlen und Text klickbar
%     bookmarksnumbered=true 	% Überschriftsnummerierung im PDF Inhalt anzeigen.
% }

% Schriftart in Captions etwas kleiner
% \setbeamerfont{caption name}{\small}

% Literaturverweise (sowohl deutsch als auch englisch)
\iflang{de}{%
\usepackage[
    backend=bibtex,		% empfohlen. Falls biber Probleme macht: bibtex
    bibwarn=true,
    bibencoding=utf8,	% wenn .bib in utf8, sonst ascii
    sortlocale=de_DE,
    style=\zitierstil,
]{biblatex}
}
\iflang{en}{%
\usepackage[
    backend=bibtex,		% empfohlen. Falls biber Probleme macht: bibtex
    bibwarn=true,
    bibencoding=utf8,	% wenn .bib in utf8, sonst ascii
    sortlocale=en_US,
    style=\zitierstil,
]{biblatex}
}
\ladeliteratur{}

\usepackage{amssymb} % Erweiterung der Symbole in Mathematikumgebung

\iflang{de}{\usepackage{icomma}} % Europäsiches Komma in Formeln

\usepackage{tabularx}
\usepackage{tabulary}

%%%%%% Custom Enviroments %%%%%%
\newenvironment{conditions}[1][\formelbeschreibung]
  {%
      #1\tabularx{\textwidth-\widthof{#1}}[t]{
      >{$}l<{$}  @{} >{${}}c<{{}$} @{} X@{}
      }%
  }
  {\endtabularx\\[\belowdisplayskip]}
%%%%%% Custom Commands %%%%%%
\newrobustcmd*{\citecompleteauthor}{\AtNextCite{\DeclareNameAlias{labelname}{given-family}}\citeauthor}

%%%%%%%%%%%%%%%%%%%%%%%%%%%%% Kopf-/Fußzeilenwechsel %%%%%%%%%%%%%%%%%%%%%%%%%%%
\setbeamertemplate{footline}
{%
    \hbox{
        \begin{beamercolorbox}[wd=\paperwidth,ht=2.5ex,dp=1ex,leftskip=.3cm,rightskip=.3cm plus1fil]{author in head/foot}%
            \titel\hfill\autor\hfill\insertpagenumber
        \end{beamercolorbox}
    }%
}

